\documentclass[UTF8]{ctexart}
\usepackage{amsmath}
\usepackage{graphicx}
\usepackage{subfig}
\usepackage{listings}
\usepackage{color}
\usepackage{hyperref}
\usepackage{geometry}
\usepackage{caption}
\usepackage[dvipsnames]{xcolor} % 更全的色系
\usepackage{listings} % 排代码用的宏包
\usepackage{svg}
\usepackage{subcaption}
%%%%%%%%%%%%%%%%%%%%%%%%%%%%%%%%%%%%%%%%
%% listings设置
%%%%%%%%%%%%%%%%%%%%%%%%%%%%%%%%%%%%%%%%

\lstset{
	language = matlab,
	backgroundcolor = \color{gray!8}, % 背景色:淡黄
	basicstyle = \small\ttfamily, % 基本样式 + 小号字体
	rulesepcolor= \color{gray}, % 代码块边框颜色
	breaklines = true, % 代码过长则换行
	numbers = left, % 行号在左侧显示
	numberstyle = \small, % 行号字体
	keywordstyle = \color{blue}, % 关键字颜色
	commentstyle =\color{green!50!black}, % 注释颜色
	stringstyle = \color{red}, % 字符串颜色
	frame = shadowbox, % 用(带影子效果)方框框住代码块
	showspaces = false, % 不显示空格
	columns = fixed, % 字间距固定
	%escapeinside={} % 特殊自定分隔符:
	morekeywords = {as}, % 自加新的关键字(必须前后都是空格)
	deletendkeywords = {compile} % 删除内定关键字;删除错误标记的关键字用deletekeywords删!
}

\geometry{a4paper, scale=0.8}
\begin{document}
\title{第二章作业}
\maketitle

\section{插值方法及原理}

\subsection{最近邻插值}  
最近邻插值是一种零阶插值方法,其基本思想是在目标图像中每个像素对应的原始图像的连续坐标处,直接选择离该坐标最近的整数位置的像素值作为插值结果。设目标图像中某点对应原始图像的连续坐标为 \((x', y')\),通过将其分别四舍五入到最近的整数坐标 \(\text{round}(x')\) 和 \(\text{round}(y')\),得到目标像素的值,即  
\[
I'(x, y) = I\big(\text{round}(x'), \text{round}(y')\big).
\]
这种方法由于只利用了单个像素的信息,因而计算量极小,适用于对实时性要求高的应用。然而,由于没有考虑周围其他像素的信息,导致在图像旋转等几何变换中容易出现块状效应、锯齿和马赛克现象,无法很好地保持图像的连续性和细节。

\subsection{双线性插值}  
双线性插值法(Bilinear Interpolation)是一种常用的二维图像插值方法,其基本思想是在水平方向和垂直方向上分别进行线性插值,以估计待插值点的像素值。该方法假设图像在局部区域内变化平缓,可以近似看作一个平面,因此在图像连续性较好的情况下能够取得较为理想的效果。

具体来说,设待插值点在原始图像中的连续坐标为 \((x', y')\)。为了进行插值,首先需要确定该点周围的四个整数坐标点,记为 \((x_1, y_1)\)、\((x_2, y_1)\)、\((x_1, y_2)\) 以及 \((x_2, y_2)\),其中  
\[
x_1 = \lfloor x' \rfloor,\quad x_2 = \lceil x' \rceil,\quad y_1 = \lfloor y' \rfloor,\quad y_2 = \lceil y' \rceil.
\]
然后定义水平和垂直方向上的偏移量为  
\[
dx = x' - x_1,\quad dy = y' - y_1.
\]

插值过程首先在水平方向上进行,对位于 \(y_1\) 和 \(y_2\) 两条水平线上分别计算插值结果:
\[
I_{y_1}(x') = I(x_1, y_1) + (x' - x_1) \big[I(x_2, y_1) - I(x_1, y_1)\big],
\]
\[
I_{y_2}(x') = I(x_1, y_2) + (x' - x_1) \big[I(x_2, y_2) - I(x_1, y_2)\big].
\]
随后,在垂直方向上对这两个中间结果进行线性插值,得到最终的像素值:
\[
I'(x, y) = I_{y_1}(x') + (y' - y_1) \big[I_{y_2}(x') - I_{y_1}(x')\big].
\]
将上述步骤合并,可以写成一个统一的表达式:
\[
I'(x, y) = (1 - dx)(1 - dy) I(x_1, y_1) + dx(1 - dy) I(x_2, y_1) + (1 - dx)dy\, I(x_1, y_2) + dx\,dy\, I(x_2, y_2).
\]

这种方法既兼顾了计算效率,也能在一定程度上保持图像的平滑性,因此被广泛应用于图像旋转、缩放等几何变换中。不过,由于它仅利用了邻近四个像素的信息,对于存在较大灰度变化或高频细节的图像,可能会引入一定程度的模糊现象。

\subsection{双三次插值}  
双三次插值是一种高阶插值方法,它考虑了目标像素周围 16 个邻近像素的信息,通过三次函数进行加权,从而在平滑性和细节保留上均表现较好。该方法基于三次卷积函数,常用的权重函数 \(w(t)\) 通常定义为  
\[
w(t) =
\begin{cases}
(a+2)|t|^3 - (a+3)|t|^2 + 1, & \text{if } |t| \le 1,\\[1mm]
a|t|^3 - 5a|t|^2 + 8a|t| - 4a, & \text{if } 1 < |t| < 2,\\[1mm]
0, & \text{if } |t| \ge 2,
\end{cases}
\]
其中参数 \(a\) 常取值为 \(-0.5\) 或 \(-0.75\),用于控制插值的锐度与平滑性。在具体计算时,设目标像素对应的原始图像连续坐标为 \((x', y')\),并定义 \(x_1 = \lfloor x' \rfloor\)、\(y_1 = \lfloor y' \rfloor\) 以及 \(dx = x' - x_1\)、\(dy = y' - y_1\)。则目标像素值通过下面的公式计算:  
\[
I'(x, y) = \sum_{i=-1}^{2}\sum_{j=-1}^{2} w(i-dx) \, w(j-dy) \, I(x_1+i, y_1+j).
\]
该方法不仅能够提供比双线性插值更为平滑的结果,还能更好地保留图像边缘和细节,特别适用于存在丰富高频信息的图像。但由于涉及更大范围的像素和更复杂的计算,其运算量明显高于前两种方法,因此在实时应用中可能受到性能限制。

\section{MATLAB 代码实现}

\begin{lstlisting}[language=matlab]
clc
clear
close all

%% 设置参数
img = imread('image.jpg');
img = rescale(img);
theta = -30;

%% 最近邻插值
tic
near = imrotate(img, theta, "nearest", "crop");
toc
figure
subplot(1,2,1)
imshow(img)
title("原始图像")
subplot(1,2,2)
imshow(near)
title("最近邻插值")

%% 双线性插值
tic
bil = imrotate(img, theta, "bilinear", "crop");
toc
figure
subplot(1,2,1)
imshow(img)
title("原始图像")
subplot(1,2,2)
imshow(bil)
title("双线性插值")

%% 双三次插值
tic
cub = imrotate(img, theta, "bicubic", "crop");
toc
figure
subplot(1,2,1)
imshow(img)
title("原始图像")
subplot(1,2,2)
imshow(cub)
title("双三次插值")

%% 对比
figure
subplot(2,2,1)
imshow(img)
title("原始图像")
subplot(2,2,2)
imshow(near)
title("最近邻插值")
subplot(2,2,3)
imshow(bil)
title("双线性插值")
subplot(2,2,4)
imshow(cub)
title("双三次插值")
\end{lstlisting}

% \section{实验结果与分析}
% 实验结果显示,对于一张640$\times$640的图像,最近邻插值计算速度最快,用时 0.003828s;双线性插值次之,用时 0.006109s;双三次插值最慢,用时 0.009154s。
% 将原图像逆时针旋转 30 度后,分别使用最近邻插值、双线性插值和双三次插值三种方法进行插值处理,得到的结果如下图所示。
% \begin{figure}[htbp]
%     \centering
%     \includesvg{fig/near.svg}
%     \caption{三种插值方法及原图的对比}
% \end{figure}

% 将同一区域放大后观察,最近邻插值计算速度快但图像质量较差,
% 容易产生锯齿。双线性插值能够显著改善图像的平滑性,但会导致一定程度的模糊。双三次插值在边缘保留和图像平滑性方面表现最佳,但计算复杂度较高。
% \begin{figure}[htbp]
%     \centering
%     \includesvg[width=0.9\textwidth]{fig/zoom.svg}
%     \caption{三种插值方法的局部放大对比}
% \end{figure}
\section{实验结果与分析}
实验结果显示,对于一张640$\times$640的图像,最近邻插值计算速度最快,用时 0.003828s;双线性插值次之,用时 0.006109s;双三次插值最慢,用时 0.009154s。
\begin{table}[htbp]
    \centering 
    \caption{插值算法处理耗时对比}
    \label{tab:time}
    \begin{tabular}{lcc}
    \hline 
    算法类型 & 计算时间(秒) & 相对耗时比例 \\ \hline 
    最近邻插值 & 0.003828 & 1.00× \\
    双线性插值 & 0.006109 & 1.60× \\
    双三次插值 & 0.009154 & 2.39× \\ \hline 
    \end{tabular}
    \end{table}

将原图像逆时针旋转 30 度后,分别使用最近邻插值、双线性插值和双三次插值三种方法进行插值处理,得到的结果如下图所示。
\begin{figure}[htbp]
    \centering 
    \subfloat[全图对比]{
    \includesvg[width=0.45\textwidth]{fig/contrast.svg}
    \label{fig:contrast}
    }
    \subfloat[局部放大]{
    \includesvg[width=0.45\textwidth]{fig/zoom.svg}
    \label{fig:zoom}
    }
    \caption{插值算法对比:(a)原始图像与三种插值结果的全图对比;(b)局部放大对比}
\end{figure}
从全图来看,最近邻插值算法呈现明显的锯齿效应,
将同一区域放大后观察,最近邻插值计算速度快但图像质量较差,
容易产生锯齿。双线性插值能够显著改善图像的平滑性,但会导致一定程度的模糊。双三次插值在边缘保留和图像平滑性方面表现最佳,但计算复杂度较高。

\end{document}



